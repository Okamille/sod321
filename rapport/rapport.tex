\documentclass{article}

% if you need to pass options to natbib, use, e.g.:
%     \PassOptionsToPackage{numbers, compress}{natbib}
% before loading neurips_2019

\usepackage[final]{neurips_2019}

% to avoid loading the natbib package, add option nonatbib:
%     \usepackage[nonatbib]{neurips_2019}

% \usepackage[latin1]{inputenc}
\usepackage[french]{babel}
\usepackage[utf8]{inputenc} % allow utf-8 input
\usepackage[T1]{fontenc}    % use 8-bit T1 fonts
\usepackage{hyperref}       % hyperlinks
\hypersetup{
  colorlinks   = true, %Colours links instead of ugly boxes
  urlcolor     = blue, %Colour for external hyperlinks
  linkcolor    = blue, %Colour of internal links
  citecolor   = red %Colour of citations
}
\usepackage{url}            % simple URL typesetting
\usepackage{booktabs}       % professional-quality tables
\usepackage{amsfonts}       % blackboard math symbols
\usepackage{nicefrac}       % compact symbols for 1/2, etc.
\usepackage{microtype}      % microtypography

\title{Optimisation discrète - Course d'avions}

\author{
  Quentin Burthier
  \\
  ENSTA Paris\\
  \texttt{quentin.burthier@ensta-paris.fr} \\
  \And
  Antoine Gauthier \\
  ENSTA Paris \\
  \texttt{antoine.gauthier@ensta-paris.fr} \\
}

\usepackage{amsmath}

\newcommand{\dij}{d_{ij}}
\newcommand{\xij}{x_{ij}}
\newcommand{\xji}{x_{ji}}
\newcommand{\Amin}{A_\text{min}}

\begin{document}

\maketitle

\begin{abstract}
  Nous étudions un problème d'optimisation semblable au voyageur de commerce.
  Nous le modélisons comme un programme linéaire en nombres entiers 0-1,
  puis nous comparons deux algorithmes de résolution basés sur des formulations
  différents des contraintes de sous-tours.
\end{abstract}
\section{Introduction}

\subsection{Notations}

On note le symbole de Kronecker $\delta$.

Une instance du problème est composée des données suivantes :
\begin{itemize}
  \item Nombre d'aérodromes : $n$
  \item Aérodrome de départ : $s$ \footnote{pour \textit{start}}
  \item Aérodrome d'arrivée : $a$
  \item Nombre d'aérodromes à parcourir : $\Amin$
  \item Nombre de régions : $m$
  \item Région de l'aérodrome $i$ : $l_i$ \footnote{pour \textit{Land}}
  \item Distance maximale de vol sans se poser : $R$
  \item Coordonnées cartésiennes de l'aérodrome $i$ : $(x_i, y_j)$
\end{itemize}

On suppose que les aérodromes se situent sur un plan et que la distance entre
deux aérodromes est la distance euclidienne. 
On note $\dij := \sqrt{(x_i - x_j)^2 + (y_i - y_j)^2}$ la distance entre 
l'aérodrome $i$ et l'aérodrome $j$.

\section{Modélisation}

\subsection{Graphe du problème}

On modélise le problème par un graphe orienté, ayant pour sommets les aérodromes.
Deux aérodromes sont reliés par une arête si et seulement si la distance entre
ces aérodromes est inférieure à la distance de vol maximale.

Le graphe est donc $G = (V, E)$ où $V = \left\{1,\dots,n\right\}$ etc
$E = \left\{(i, j) \in V^2, i \neq j : d_ij \leq R\right\}$.

\subsection{Objectif}

On cherche à minimiser la distance parcourue, soit
\begin{align}
  \mathcal{Z} = \min \sum_{i,j : (i, j) \in E} \xij \dij
\end{align}
où $\xij \in \left\{0, 1\right\}$.

\subsection{Contraintes du problème}

\paragraph{Unicité des visites} Chaque aérodrome est visité au plus une fois, 
sauf les aéroports d'arrivée et de départ qui sont visités exactement une fois.
\begin{subequations}
\begin{align}
  \forall i \notin \left\{s, a\right\},\;
    \sum_{j: (i, j) \in E} \xij \leq 1 \\
  \forall i \notin \left\{s, a\right\},\;
    \sum_{j: (j, i) \in E} \xji \leq 1
\end{align}
\end{subequations}

\paragraph{Connexité}
\begin{subequations}
  \label{cons:connex}
L'avion doit partir de $s$
\begin{align}
  \sum_{j: (s, j) \in E} x_{sj} & = 1 \\
  \sum_{j \neq a: (j, s) \in E} x_{js} & = 0\\
  x_{as} & = \delta_{as} \label{cons:s_eq_a}
\end{align}
et arriver en $a$
\begin{align}
  \sum_{j: (j, a) \in E} x_{ja} & = 1 \\
  \sum_{j \neq a : (a, j) \in E} x_{aj} & = 0 \\
\end{align}
La contrainte \ref{cons:s_eq_a}
autorise que les aérodromes d'arrivée et de
départ soient identiques. Cela suppose que $d_{as} \leq R$,
sinon le problème n'est par réalisable.
\medbreak

Un avion repart d'un aérodrome seulement si et seulement 
si il s'est posé à cet aérodrome.
\begin{align}  
  \forall i \notin \left\{s, a\right\},\; 
    \sum_{j: (j, i) \in E} \xji = \sum_{j : (i, j) \in E} \xij
\end{align}
\end{subequations}

\paragraph{Nombre minimum d'aérodromes visités} L'avion doit visiter au moins
$\Amin$ aérodromes.
\begin{align}
  \sum_{i, j: (i, j) \in E} \xij \geq \Amin - 1 + \delta_{sa}
\end{align}
Ici encore $\delta_{sa}$ permet de tenir compte du cas où $s = a$ et que
la solution est un cycle.

\paragraph{Diversité des régions} La contrainte de visiter au moins une fois
chaque région une fois s'écrit, en notant
$\mathcal{A}_l := \left\{ i : l_i = a \right\}$
l'ensemble des aéroports appartenant à la région $a$, et $\mathcal{L}$
l'ensemble des régions comportant au moins un aérodrome.
\begin{align}
  \forall l \in \mathcal{L} \setminus \left\{l_s, l_a\right\},\;
  \sum_{i \in \mathcal{A}_l: (i, j) \in E} \xij \geq 1
\end{align}
On a supprimé les contraintes sur les régions $r_s$ et $r_a$, 
qui sont redondantes avec les contraintes d'arrivée et de départ.

\subsection{Contraintes d'élimination des sous-tours}

Il nous reste à supprimer la possibilité d'avoir solutions composées de plusieurs
cycles.

\begin{subequations}
\paragraph{Nombre polynomial de contraintes}

\begin{align}
  & \forall i \neq s, j\neq a: (i, j) \in E,\; 
    u_j \geq u_i - n (1 - \xij) \\
  &  u_i \in \mathbb{Z}
\end{align}
\end{subequations}
Ceci force $u_i > u_j$ quand l'avion va de $i$ à $j$, et interdit donc les cycles.

\paragraph{Nombre exponentiel de contraintes}
On peut aussi interdire les sous-tours au sein de tous les sous-graphes.
\begin{align}
  \forall S \subset G,\;
  \sum_{i \in S, j \in S} \xij \leq |S| - 1 \label{cons:subtours}
\end{align}
Cependant, le nombre de sous-graphe croit exponentiellement avec la taille du
graphe.

Il faut donc générer donc des contraintes à chaque nœud de
l'algorithme \textit{branch \& cut}.
Pour cela on détecte d'abord les contraintes violées parmi (\ref{cons:subtours}),
puis on ajoute ces contraintes au modèle.

\section{Étude numérique}

\end{document}
