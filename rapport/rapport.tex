\documentclass{article}

% if you need to pass options to natbib, use, e.g.:
%     \PassOptionsToPackage{numbers, compress}{natbib}
% before loading neurips_2019

% ready for submission
% \usepackage{neurips_2019}

% to compile a preprint version, e.g., for submission to arXiv, add add the
% [preprint] option:
%     \usepackage[preprint]{neurips_2019}

% to compile a camera-ready version, add the [final] option, e.g.:
     \usepackage[final]{neurips_2019}

% to avoid loading the natbib package, add option nonatbib:
%     \usepackage[nonatbib]{neurips_2019}

% \usepackage[latin1]{inputenc}
\usepackage[french]{babel}
\usepackage[utf8]{inputenc} % allow utf-8 input
\usepackage[T1]{fontenc}    % use 8-bit T1 fonts
\usepackage{hyperref}       % hyperlinks
\usepackage{url}            % simple URL typesetting
\usepackage{booktabs}       % professional-quality tables
\usepackage{amsfonts}       % blackboard math symbols
\usepackage{nicefrac}       % compact symbols for 1/2, etc.
\usepackage{microtype}      % microtypography

\title{Optimisation discrète - Course d'avions}

% The \author macro works with any number of authors. There are two commands
% used to separate the names and addresses of multiple authors: \And and \AND.
%
% Using \And between authors leaves it to LaTeX to determine where to break the
% lines. Using \AND forces a line break at that point. So, if LaTeX puts 3 of 4
% authors names on the first line, and the last on the second line, try using
% \AND instead of \And before the third author name.

\author{%
  Quentin Burthier
  % \thanks{Use footnote for providing further information
  %   about author (webpage, alternative address)---\emph{not} for acknowledging
  %   funding agencies.}
  \\
  ENSTA Paris\\
  \texttt{quentin.burthier@ensta-paris.fr} \\
  \And
  Antoine Gauthier \\
  ENSTA Paris \\
  \texttt{antoine.gauthier@ensta-paris.fr} \\
}

\usepackage{amsmath}

\newcommand{\dij}{d_{ij}}
\newcommand{\xij}{x_{ij}}
\newcommand{\xji}{x_{ji}}
\newcommand{\Amin}{A_\text{min}}

\begin{document}

\maketitle

\begin{abstract}
  The abstract paragraph should be indented \nicefrac{1}{2}~inch (3~picas) on
  both the left- and right-hand margins. Use 10~point type, with a vertical
  spacing (leading) of 11~points.  The word \textbf{Abstract} must be centered,
  bold, and in point size 12. Two line spaces precede the abstract. The abstract
  must be limited to one paragraph.
\end{abstract}

\section{Introduction}

\subsection{Notations}

Une instance du problème est composée des données suivantes :
\begin{itemize}
  \item Nombre d'aérodromes : $n$
  \item Aérodrome de départ : $k$
  \item Aérodrome d'arrivée : $l$
  \item Nombre d'aérodromes à parcourir : $\Amin$
  \item Nombre de régions : $m$
  \item Région de l'aérodrome $i$ : $r_i$
  \item Distance maximale de vol sans se poser : $R$
  \item Coordonnées cartésiennes de l'aérodrome $i$ : $(x_i, y_j)$
\end{itemize}

On suppose que la distance entre deux aérodromes est la distance euclidienne, et 
on note $\dij := \sqrt{(x_i - x_j)^2 + (y_i - y_j)^2}$ la distance entre 
l'aérodrome $i$ et l'aérodrome $j$.

\section{Modélisation}

\subsection{Objectif}

On cherche à minimiser la distance parcourue, soit
\begin{align}
  \mathcal{Z} = \min \sum_{i, j} \xij \dij
\end{align}
où $\xij \in \left\{0, 1\right\}$.

\subsection{Contraintes du problème}

\paragraph{Unicité des visites} Chaque aérodrome est visité au plus une fois, 
sauf les aéroports d'arrivée et de départ qui sont visités exactement une fois.
\begin{subequations}
\begin{align}
  \forall i \notin \left\{k, l\right\},\; \sum_j \xij \leq 1 \\
  \forall i \notin \left\{k, l\right\},\; \sum_j \xji \leq 1
\end{align}
\end{subequations}

\paragraph{Connexité}
\begin{subequations}
L'avion doit partir de $k$
\begin{align}
  \sum_j x_{kj} & = 1 \\
  \sum_j x_{jk} & = \delta_{kl}
\end{align}
et arriver en $l$
\begin{align}
  \sum_j x_{lj} & = \delta_{kl} \\
  \sum_j x_{jl} & = 1
\end{align}
On autorise ainsi que les aérodromes d'arrivée et de départ soient identiques.
\medbreak

Un avion repart d'un aérodrome seulement si et seulement 
si il s'est posé à cet aérodrome.
\begin{align}  
  \forall i \notin \left\{k, l\right\},\; \sum_j \xji = \sum_j \xij
\end{align}
\end{subequations}

\paragraph{Nombre minimum d'aérodromes visités} L'avion doit visiter au moins
$\Amin$ aérodromes :
\begin{align}
  \sum_{i,j} \xij \geq \Amin
\end{align}

\paragraph{Diversité des régions} La contrainte de visiter au moins une fois
chaque région une fois s'écrit, en notant $I_r := \left\{ i : r_i = r \right\}$
l'ensemble des aéroports appartenant à la région $r$, et $R$ l'ensemble des 
régions comportant au moins un aérodrome,
\begin{align}
  \forall r \in R \setminus \left\{0, r_k, r_l\right\},\;
  \sum_{i \in I_r} \xij \geq 1
\end{align}
On a supprimé les contraintes sur les régions $r_1$ et $r_l$, 
qui sont redondantes avec \textbf{FILLME}

\paragraph{Distance de vol maximale} Un avion peut parcourir une distance d'au plus
$R$ sans se poser. Cela se traduit par
\begin{align}
  \forall i, j : i \neq j,\; \xij \dij \leq R
\end{align}

\subsection{Contraintes d'élimination des sous-tours}

Il nous reste à supprimer la possibilité d'avoir solutions composées de plusieurs
cycles.

\begin{subequations}
\paragraph{Nombre polynomial de contraintes}

\begin{align}
  \forall i,j : i \neq k, j \neq k,\; u_j \geq u_i - n (1 - \xij)
\end{align}
Cela force $u_i > u_j$ pour $i$ et $j$ connectés, et interdit donc les cycles 
(sauf pour $i = k$ ou $j = k$, ce qui permet de revenir en $k$ si $k = l$).

\paragraph{Nombre exponentiel de contraintes}

\begin{align}
  \forall S \subset (V \setminus \left\{l\right\})
                \cup
                (V \setminus \left\{r\right\}),
          |S| > 1,\;
  \sum_{i \in S, j \in S} \xij \leq |S| - 1,
\end{align}

\end{subequations}

\section{Étude numérique}

\end{document}
